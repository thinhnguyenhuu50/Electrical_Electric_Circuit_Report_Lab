\section{Tổng quan đề tài}
\subsection{Mục tiêu}
Trong bài tập lớn này, nhóm sử dụng Altium Designer để thiết kế và bố trí mạch điện
tổng hợp nhiều kiến thức đã học trong môn học này.

Bao gồm:

\begin{itemize}
    \item Cách thiết kế công tắc đầu vào kỹ thuật số (digital input switches).
    \item Cách sử dụng diode để tránh đoản mạch.
    \item Cách sử dụng bóng bán dẫn tạo ra tín hiệu dòng điện cao (transistor generating a high current signal).
    \item Cách sử dụng opamp làm bộ đệm (buffer) và bộ lọc thông thấp (a low-pass filter) trước khi đọc tín hiệu ADC.
    \item Cách kết nối tất cả tín hiệu đầu vào và đầu ra với bộ vi điều khiển (micro-controller).
\end{itemize}

\subsection{Thông số kỹ thuật}
Trong dự án này, mục tiêu của nhóm là thiết kế một mạch có khả năng:

\begin{itemize}
    \item Đo dòng điện của tín hiệu AC 220V.
    \item Đặt địa chỉ (address) để phân biệt với các mạch tương tự khác, tối đa 16.
    \item Đo dòng điện tối đa lên tới 5A hoặc lên tới 10A.
    \item Gửi dữ liệu đến một cổng qua RS485 hoặc Wifi hoặc Bluetooth.
    \item Hiển thị trên LED 7 đoạn dùng IC 74HC595.
\end{itemize}

\subsection{Giải pháp}
Để đáp ứng được yêu cầu trên, một trong những giải pháp mà chúng ta có thể nghĩ tới đó là:

\begin{itemize}
    \item Chúng ta sẽ sử dụng cảm biến dòng điện [1] (current sensor) để đo dòng điện của tín hiệu AC. Cảm biến
    nên hỗ trợ đo tới 5A hoặc lên tới 10A. Có nhiều cảm biến hiện tại
    có sẵn trên thị trường, nhóm sẽ sử dụng TA12 [2] cho dòng điện tối đa 5A
    và TA17 cho tối đa 10A. Chúng rẻ và dễ sử dụng.
    \item Chúng ta sẽ sử dụng 4 công tắc trượt (slide switches) để đặt địa chỉ bảng (board address).
    \item Chúng ta sẽ sử dụng một IC có thể chuyển đổi từ tín hiệu UART sang tín hiệu RS485 [3] để truyền dữ liệu qua RS485.
    \item Chúng ta sẽ sử dụng bo mạch vi điều khiển (MCU), cụ thể là ESP32-WROOM-32 [4] làm
    bộ xử lý chính. ESP32-WROOM-32 là MCU WiFi và Bluetooth chung, mạnh mẽ
    mô-đun phù hợp với nhiều ứng dụng IoT.
\end{itemize}

Nhóm sẽ liệt kê tất cả các phần cần thiết cho mạch này. Dựa vào giải pháp trên,
mạch nên bao gồm:
\begin{itemize}
    \item Một nguồn điện đầu vào có dải điện áp 5V - 36V.
    \item Bộ điều chỉnh 3,3V để cấp nguồn cho mô-đun ESP32-WROOM-32.
    \item Một bo mạch vi điều khiển ESP32-WROOM-32.
    \item Cảm biến TA12/TA17 hỗ trợ dòng điện lên tới 5A hoặc lên đến 10A.
    \item Công tắc trượt cho địa chỉ bảng.
    \item Đèn LED hiển thị trạng thái.
    \item Mạch RS485.
    \item Một số tụ lọc nhiễu.
\end{itemize}


